\documentclass{article} % For LaTeX2e
\usepackage{nips14submit_e,times}
\usepackage{hyperref}
\usepackage{url}
%\documentstyle[nips14submit_09,times,art10]{article} % For LaTeX 2.09

\title{WikiCat: Automatic Categorization of Wikipedia Pages into Categories}
\author{Myung-ha Jang \& John Foley\\
  Center for Intelligent Information Retrieval\\
College of Information and Computer Sciences\\
University of Massachusetts Amherst\\
\texttt{\{mhjang,jfoley\}@cs.umass.edu}
}

\newcommand{\fix}{\marginpar{FIX}}
\newcommand{\new}{\marginpar{NEW}}

%\nipsfinalcopy % Uncomment for camera-ready version

\begin{document}


\maketitle

\begin{abstract}
The abstract should be clever, exciting and short.
\end{abstract}

\section{Introduction}
Researchers love wikipedia~\cite{wang2014concept,banerjee07,gabrilovich2007computing,meij2012adding,de2014taxonomic, pohl2012classifying}.

The winner of the 2014 Kaggle challenge~\cite{puurula2014kaggle} used an ensemble of many classifiers.

\bibliographystyle{abbrv}
\small\bibliography{cites}
\end{document}
